\documentclass[10pt]{article}
\usepackage[margin=1.0in]{geometry}
\setlength\parindent{0pt}
\pagenumbering{gobble}

\usepackage{paracol}
\setlength{\columnsep}{4em}

\usepackage{enumitem}

\usepackage{xurl}
\usepackage{hyperref}
\hypersetup{
    colorlinks=true,
    urlcolor=blue,
}

\newenvironment{itemize*}
{\begin{itemize}[leftmargin=*]
    \setlength{\itemsep}{0pt}
    \setlength{\parskip}{0pt}}
{\end{itemize}}

\begin{document}

% document should have 2 columns
\noindent
\parbox[t]{0.5\textwidth}{
    {\sffamily\Huge Josie Thompson}\medskip\\
    6502 27th Ave NE\\
    Seattle, WA 98115\\
    (253) 227-0840\\
    josiest@cs.washington.edu
}
\vspace{12pt}
\hrule

\begin{paracol}{2}
\section*{Education Experience}
\textbf{Goal:} To earn a PhD in mathematics
\begin{description}
\item[University of Washington]~\\
    Seattle, Washington\\
    Working toward bachelor's\\
    B.S. in computer science\\
    B.A. in mathematics
\item[Tacoma Community College]~\\
    Tacoma, Washington\\
    A.S. in electrical engineering
\end{description}
\section*{Work Experience}
\begin{description}
\item[Lab Assistant]~\\
    University of Washington Herbarium\\
    September 2019 - December 2019\\
    \url{https://github.com/josiest/Flora-Data-Extraction}
\end{description}
\begin{itemize*}
\item Used regular expressions to extract geographical information from text
\item Used python to transform text data from pdfs to csv data
\end{itemize*}

\begin{description}
\item[Internationally Certified Tutor]~\\
    Computer science and mathematics\\
    July 2018 - August 2019
\end{description}
\begin{itemize*}
\item Helped students understand new concepts
\item Encouraged critical thinking by asking leading questions
\end{itemize*}

\begin{description}
\item[Supplementary Instruction (SI) Leader]~\\
    Computer science\\
    March 2019 - August 2019
\end{description}
\begin{itemize*}
\item Planned SI sessions that encouraged collaborative learning
\item Facilitated group discussion to reflect on new material
\item Designed creative activities that reinforced class material
\end{itemize*}

\switchcolumn
\section*{Programming Experience}
\begin{description}
\item[Skilled in technologies]
\end{description}
\begin{itemize*}
\item Python, C/C++, Java, Mathematica, Matlab
\end{itemize*}

\begin{description}
\item[Hax and mapgen]~\\
    \url{https://github.com/josiest/hax}\\
    \url{https://github.com/josiest/mapgen}\\
    Small libraries for working with hexagonal maps
\end{description}
\begin{itemize*}
\item Implemented algorithms for working with unique mathematical norms in C++
\item Wrote well-covered tests
\item Wrote clear documentation
\end{itemize*}

\begin{description}
\item[Pygtails Library]~\\
    \url{https://pygtails.readthedocs.io/en/latest/}\\
    Event-handler for Python's Pygame library
\end{description}
\begin{itemize*}
\item Implemented event-handling interface
\item Used Sphinx and reStructuredText to write and compile original tutorials and clear
      documentation
\item Published module to Python Package Index
\end{itemize*}

\begin{description}
\item[404]~\\
    \url{https://github.com/josiest/404}\\
    Virtual art piece on the concept of Utopia
\end{description}
\begin{itemize*}
\item Used lua and love2d to create a new media art piece reflecting on the definition of Utopia
\item Used mathematical models of electricity and friction to emulate an imagined force
\end{itemize*}

\section*{Extracurricular Activities}

\begin{description}
\item[Programming Club]~\\
    Treasurer\\
    January 2018 - June 2018
\end{description}
\begin{itemize*}
\item Presented a lecture on programming languages
\item Held decisions on club purchases
\item Helped in organizational decisions
\end{itemize*}
\end{paracol}

\end{document}
