\documentclass[10pt]{article}
\usepackage[margin=0.75in]{geometry}
\usepackage[utf8]{inputenc}

\usepackage[sfdefault,extralight]{FiraSans}
\usepackage[T1]{fontenc}
\renewcommand*\oldstylenums[1]{{\firaoldstyle #1}}

\setlength\parindent{0pt}
\pagenumbering{gobble}

\usepackage{paracol}
\setlength{\columnsep}{4em}

\usepackage{calc}
\usepackage{enumitem}

\usepackage{xurl}
\usepackage{hyperref}
\hypersetup{
    colorlinks=true,
    urlcolor=blue,
}

\newenvironment{itemize*}
{\begin{itemize}[leftmargin=*]
    \setlength{\parskip}{0.5pt}}
{\end{itemize}}

\begin{document}

% document should have 2 columns
\begin{paracol}{2}
\noindent
\parbox[t]{0.5\textwidth}{
    {\sffamily\Huge Josie Thompson}\medskip\\
    dot.slash.josie@gmail.com \\
    Seattle, WA 98122\\
    (253) 227-0840
}
\switchcolumn

\section*{Education History}

\begin{description}
\itemsep -0.5em
\item[B.S. in Computer Science] \hfill \\
    University of Washington (2022)\\
    Seattle, Washington
\end{description}

\begin{description}
\itemsep -0.5em
\item[A.S. in Electrical and Computer Engineering] \hfill \\
    Tacoma Community College (2019)\\
    Tacoma, Washington
\end{description}


\textbf{Portfolio:} \url{https://josiest.github.io}

\end{paracol}
\vspace{12pt}
\hrule
\section*{Research Experience}
\begin{description}[leftmargin=!,
                    labelwidth=\widthof{\bfseries SOFA Collider Project}]

\item[SOFA Collider Project] DXARTS | University of Washington \hfill 
    January 2022 - August 2022
\end{description}
\begin{itemize*}
\item Rapidly prototyped a library for processing multi-directional HRTF audio data with SuperCollider
\item Formed creative solutions to interfacing between languages in order to use existing tools to accelerate project development
\item Wrote tools to read and write data to form a standard HRTF format into the SuperCollider runtime environment
\end{itemize*}
\vspace{10pt}

\begin{description}[leftmargin=!,
                    labelwidth=\widthof{\bfseries Research Assistant}]

\item[Research Assistant] Robot Learning Lab | Paul Allen School of Computer
    Science \hfill 
    January 2022 - August 2022
\end{description}
\begin{itemize*}
\item Read through many recent peer-reviewed papers in order to have a good grasp of the problem at hand
\item Ipmlemented complex algorithms in order to further push knowledge of concepts in the research papers
\item Used advanced machine learning libraries in python to conduct experiments
\item Wrote extensive documentation and established organized project structure in order to maintain a practical environment
\end{itemize*}

\section*{Work Experience}
\begin{description}[leftmargin=!, labelwidth=\widthof{\bfseries Tutor}]
\item[Tutor] Seattle Central College \hfill 
    January 2024 - Present
\end{description}
\begin{itemize*}
\item Asked leading questions to help students come to their own conclusions
\item Helped students with problems in-class as well as at tutoring center
\item Researched and compiled resources and reference sheets to help students solve problems
\end{itemize*}
\vspace{10pt}

\begin{description}
\item[Unreal Gameplay Engineer] Timberline Studio Inc.
    \hfill April 2022 - November 2023
\end{description}
\begin{itemize*}
\item Solved problems using different mathematical disciplines from 3D spatial reasoning to graph theory
\item Communicated goals, problems and research into solutions on a daily basis
\item Published in-depth documentation on how to use custom tools complite with tutorials, pictures and videos
\item Wrote designer-friendly interfaces from C++ code into blueprint scripts
\end{itemize*}
\vspace{10pt}

\section*{Work Experience (cont.)}
\begin{description}[leftmargin=!,
                    labelwidth=\widthof{\bfseries Teaching Assistant}]

\item[Teaching Assistant] Paul Allen School of Computer Science \hfill 
    March 2021 - December 2021\\
    Software Design and Implementation
\end{description}
\begin{itemize*}
\item Used console applications to publish course assignments within a tight schedule
\item Resolved dozens of special case problems with student homework submissions per week
\item Managed organization for grading assignments for a class of nearly three hundred students
\item Graded 15-30 students each week as part of my normal TA duties
\item Held Office Hours and helped teach supplemental sections for students to
    broaden the understanding
\end{itemize*}
\vspace{10pt}

\begin{description}[leftmargin=!,
                    labelwidth=\widthof{\bfseries Grader}]
\item[Grader] University of Washington \hfill
    September 2020 - March 2021\\
    Advanced Linear Algebra
\end{description}
\begin{itemize*}
\item Graded 2-3 problems on an assignment each week in a class of over 100 students
\item Designed rubrics for grading based off solutions and notes from the Instructor
\item Gave feedback on proofs written by students, with varying strictness per problem
\end{itemize*}
\vspace{10pt}

\begin{description}[leftmargin=!,
                    labelwidth=\widthof{\bfseries Lab Assistant}]
\item[Lab Assistant] Burke Herbarium | University of Washington \hfill
    September 2019 - December 2019\\
\end{description}
\begin{itemize*}
\item Used regular expressions to extract notes about flora from text
\item Used python to transform text data from text in pdf form to csv data
\item Designed a console interface to easily process multiple pdfs at once
\end{itemize*}
\vspace{10pt}


\begin{description}[leftmargin=!,
                    labelwidth=\widthof{\bfseries Internationally Certified Tutor}]

\item[Internationally Certified Tutor] Tacoma Community College \hfill
    July 2018 – August 2019\\
    Math and Computer Science \hfill Level 2 Certification
\end{description}

\begin{itemize*}
\item Worked as a tutor in two different learning centers to help
      with all Math and Computer Science classes offered
\item Helped study groups and individual students in one center on an on-call basis
\item Took scheduled sessions at another center where students could have a
      longer session to work on their problems
\item Used leading questions when helping students to give them the tools to
      solve similar problems on their own
\end{itemize*}
\vspace{10pt}

\begin{description}[leftmargin=!,
                    labelwidth=\widthof{\bfseries Supplementary Instruction Leader}]

\item[Supplementary Instruction Leader] Tacoma Community College \hfill
    March 2019 – August 2019\\
    Computer Science
\end{description}
\begin{itemize*}
\item Worked with students in-class to help with activities and questions about
      Java and general programming concepts
\item Designed lesson plans for supplemental instruction that
      encouraged collaborative learning and critical thinking
\item Included fun and interesting activities to keep students engaged in the material
\end{itemize*}
\end{document}
